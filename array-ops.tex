\documentclass{article}
\usepackage[utf8]{inputenc}

\title{Operations for arrays in Cogent}

\usepackage{amsmath}
\usepackage{amssymb}
\usepackage{amsthm}
% \usepackage{semantic}
\usepackage{mathpartir}

\newtheorem{theorem}{Theorem}
\newtheorem{remark}[theorem]{Remark}
\newtheorem{question}[theorem]{Question}

\begin{document}

\maketitle

Currently, only one hole (= taken index) is allowed in arrays, rather than
multiple ones. This avoids the need for checking that two general symbolic
indices differ (to prevent taking twice the same index).
Note that checking equality of symbolic indices is still required for Put, but
the trivial syntactic equality test is, I guess, already strong enough for
common usage.

We say that that an index is \textbf{present} if the hole is (provably) not at its
position.

\begin{remark}
  The led driver explicitely involves writable arrays of non-linear (discardable and sharable) elements.
  Through the bang operation, it also involves read-only arrays of non-linear
  elements.
\end{remark}
\begin{remark}
 Constant sized arrays can be (non-efficiently) copmiled to records. Accessing some symbolic index 
 would be translated by explicitely pattern-matching on the value of the index.
\end{remark}
Here are the available operations for arrays:

\begin{description}
\item[index] (similar to Member) for discardable arrays: return the $n^{th}$
  element of an array, where $n$ is a present symbolic index
 which is provably smaller than the array size (this condition is to be
 implemented later with refinement types).
 \item[take] (similar to Take for records, not used in the led driver), executes
   a continuation provided with
   \begin{itemize}
   \item 
     the $n^{th}$ element of an array (if the
     array has no holes\footnote{Take could be allowed for
       arrays with holes if the type is sharable. In this case, it could be enforced (or
       not?) that the hole remains unchanged in the continuation.}), for $n$ a symbolic index provably smaller than the array
     size,
     \item the array where the hole is set at $n$, unless the type is sharable,
       in which case the hole may be left unset.
   \end{itemize}
 \item[put] updates a non-readonly array anywhere, as long as the type is discardable.
  \item[put'] (not used in the led driver) updates a non-readonly array at a taken symbolic index (equality between the
    taken index and the put index can be left syntactic, or refined later using
    refinement types), and returns the updated array
    \item[fold] (not used in the led driver) for discardable arrays. There could also be a version returning the
      array, or executing a continuation, for non-discardable sharable
      arrays\footnote{For non-discardable arrays, sharability is required, otherwise
        all the indices should be marked as taken, but we only allow arrays with
      one hole.}.
    \item[map2] takes two non-readonly arrays without holes, a function taking
      two arguments and returning their updated values, and returns the pair of
      updated arrays.
\end{description}


\end{document}