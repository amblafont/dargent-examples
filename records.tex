\documentclass{article}
\usepackage[utf8]{inputenc}

\title{Record operations in Cogent}

\usepackage{amsmath}
\usepackage{amssymb}
\usepackage{amsthm}
% \usepackage{semantic}
\usepackage{mathpartir}

\newtheorem{theorem}{Theorem}
\newtheorem{remark}[theorem]{Remark}
\newtheorem{question}[theorem]{Question}

\begin{document}

\maketitle

\section{Permissions}
A type may have multiple permissions among $S$ (sharable), $D$ (discardable), $E$ (escapable).
Sharable elements of the context can be duplicated, whereas discardable
elements can be dropped.

We deem \textbf{non-linear} any type which is sharable and discardable.
Any type can become non-linear through the bang type operation. 

We call \textbf{writable} a type which allows in-place modifications. For
example, unboxed types are not writable in this sense, but writable records are.

\begin{remark}
\label{r:S-D-non-lin}  
  Currently, any sharable or discardable type is non linear.
  However, the distinction between sharable and discardable is maintained in the view of future extensions.
\end{remark}
\begin{remark}
 \label{r:S-D-non-writable} Non-linear types currently coincide with non-writable (in the
 sense of in-place updatable) types. This coincidence is crucial for the typing rule of
 Member, as we will see.
\end{remark}


\section{Records}
Each record type specifies a mode $r$ (read-only), $w$ (writable), $u$ (unboxed).
The permissions of a record type are a subset of the permissions of its mode,
recalled below:
\[
  r:_\kappa {D,S}\qquad
  w:_\kappa {E}\qquad
  u:_\kappa {D,S,E}
\]
More exactly, a record type has some permission if its mode has it and each
(non-taken) field type has it (Cf. rule $\mathsf{KRec}$ of the ICFP paper).
Banging $w$ yields $r$.

\begin{remark}
 The update static semantics (see rule $\text{RL}_2$ of the ICFP paper) relies on the
 assumption that read-only records are always sharable, that is, their present
 fields are always sharable. The update static semantics is thus laxer (i.e. it
 types more).
\end{remark}

\subsection{Taken fields}
In order to enforce linearity, the type system keeps track of which fields
have been already taken in a record (in order to prevent accessing twice the
same field). Vincent noted however that it is unnecessary to keep track of this
information for the subfields of a taken field.

% In particular, if a record is non linear, then all its subfi

\subsection{Operations on records}
The record operations consist in updating or getting some field.

The most obvious get operation is called member: it returns the field value
of some record. However, this member operation is too restrictive: 
it does not allow to access the different fields of a non sharable record.
This is the motivation for the alternative Take operation,
which returns (in some sense) the record.
\subsubsection{Member}
The member operation allow to
access a (present) field
of a
sharable record.

\paragraph{Critique} This typing rule does not match Liam's PhD thesis in two
respects:
\begin{itemize}
\item it involves the sharable kind rather than discardable kind (although currently
  both are equivalent, see Remark~\ref{r:S-D-non-lin})
  \item the whole record is required to have the kind, rather than the record
    without the involved field.
    The reason for enforcing that the involved field has the kind
    is, I guess, to prevent accessing a writable field of a readonly
    record\footnote{This is unwanted because then the writable subrecord could
      be in-place updated, although it was part of a readonly record.}. However, this
    relies on the coincidence that in the current system, non-linear types
    coincide with non-writable types (Remark~\ref{r:S-D-non-writable}), which
    might not be true in the future.
\end{itemize}
\paragraph{Suggestions}
First, we could require that the record without the involved field is discardable.
Then, this opens the possibility of accessing a writable field of a readonly
record.
Here are some ways of forbiding this, in the order of increasing required amount of work (according to me):
\begin{enumerate}
\item introduce a well-formed predicate on types to prevent nesting a writable record in a readonly
  record\footnote{Although such types are constructible in the ICFP paper, no
    operation is available for them according to the ICFP paper, and
    they are not constructible with the implemented surface language.}.
  \item Introduce a new kind $W$ for writable types, and forbids accessing a writable
    field of a readonly record.
\end{enumerate}
\begin{remark}
 Another suggestion consists in banging the return type of member when the record is readonly.
 However, Vincent is concerned that this does not match the bang supposed
 behaviour, which might be problematic for future extensions.
 I trust him.
\end{remark}
\subsection{Take}
    access a (present) field of a non-read-only record, with a continuation where
    the field is marked as taken. In the case the field is sharable,
    this marking is not necessary, although still possible.
    \begin{remark}
      In these remarks, I assume that Member has the corrected typing rule
      suggested above.
      \begin{itemize}
        % \item For a sharable field of a non-read-only record, both take operations are
        %   possible (i.e., marking the field as taken or keeping it as present in the continuation).
      \item Instead of having a continuation, Take could return the field and the record 
        with the field marked as Taken.
      % \item For accessing a field of an unboxed record whose other
      %   fields are discardable, both take and member operations are available.
      %   \begin{itemize}
      %   \item Take can be used to mark the field as Taken.
      %   \item There is no essential difference in using Member or a Take which
      %     does not mark the field as Taken.
      %   \end{itemize}
      % \item Take could be allowed for sharable read-only records as well.
        \end{itemize}
        \end{remark}
        \paragraph{Critique}
        Member can be thought of as a simplified version of Take. Its status is
        not very clear, because in some specific case it is redundant with Take,
        and in others it is not.
        % For the sake
        % of concision, it would make sense to have only Take.
\paragraph{Suggestion}
It would make sense to extend Take so that it covers the Member case as well.
Then Member could be thought merely as an additionnal convenience.
In this case, we just have to remove the non-readonly constraint (in this case,
some, we should prevent accessing a writable field of a read-only record, as
following suggestions of the previous section on Member).
    \subsubsection{Put}
    updating a taken or\footnote{This 'or' is not exclusive.} discardable field of a non-read-only record, returning the
      updated record where the field is marked as present. 
      \subsubsection{Remarks}
\begin{remark}
    As mentioned above, read-only records with at least one non-sharable field are useless, as no
    operation is available for them, although it would be harmeless to allow
    take or member for their sharable fields.
  \end{remark}
  \begin{remark}
    One may consider allowing Take and Member for sharable taken fields.
    This would be safe if taken were not used to mark uninitialized
    fields\footnote{%
      The return type of abstract functions can contain taken annotations,
      meaning that some fields are not initialized.
    }.
    But ignoring this issue (for example, introducing a proper uninitialized
    flag),
    one could then consider removing the possibility of keeping a sharable field as present for Take.
   But, in this case, you could not take a sharable field and return the
   original record without taken annotation, so some subtyping rule should be
   added to make the taken annotation for sharable fields irrelevant, or, as a
   hack, you could re-put the taken value (or introduce an explicit cast operation).
  \end{remark}
\end{document}
